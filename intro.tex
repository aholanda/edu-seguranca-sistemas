\lecturetitle{Introdução}{\course}
\frame{\maketitle}

\lecture{A segurança da informação e os aspectos físico, lógico e humano}{info}

\begin{frame}{Análise da Segurança de Sistemas~\small\cite{ross2008}}

  Uma boa Engenharia de Segurança de Sistemas requer a coesão de
  4 aspectos:

  \begin{description}[<+-| alert@+>]\setbeamercovered{transparent}
    \item[Políticas:] regras a serem seguidas ou implementadas.
    \item[Mecanismos:] ferramentas para implementar estas regras, por
      exemplo, criptografia e controle de acesso.
    \item[Confiabilidade:] o quanto de confiança deve ser colocado em cada mecanismo.
    \item[Incentivo:] as pessoas devem ser incentivadas ao cumprimento das políticas
      para garantir a confiabilidade dos mecanismos (Engenharia Social).
  \end{description}
\end{frame}

\begin{frame}{Exemplo 1~\small\cite{ross2008}}\small
  Um {\bf banco} possui uma gama de sistemas que exigem um alto nível de segurança:

  \begin{enumerate}[<+-| alert@+>]\setbeamercovered{transparent}
  \item O cerne das operações bancárias é o sistema de escrituração
    contábil. Ele armazena as informações do cliente e as transações.
    Há várias políticas, dentre elas, cada crédito deve corresponder
    a um débito, nada pode ser apagado, em altas transações duas ou mais
    pessoas devem autorizar, dentre outras. Os ataques normalmente ocorrem
    internamente, 1\% dos bancários são despedidos a cada ano.
  \item O caixa eletrônico utiliza autenticação e criptografia e está
    sujeito a ataques externos e internos extremamente difíceis de
    serem identificados. O uso de criptografia nos caixas ajudou a
    estabelecer padrões criptográficos.
  \item A internet, {\em website} ou dispositivos móveis, apesar do
    uso de controle de acesso e criptografia, tornou um fonte de vulnerabilidade
    através de ataques conhecidos como {\em phishing}, que são difíceis
    de mitigar pois usam fatores psicológicos.
  \end{enumerate}
\end{frame}

\begin{frame}{Exemplo 2~\small\cite{ross2008}}\small

  Sistema biomédico de um {\bf hospital}:
  
  \begin{enumerate}[<+-| alert@+>]\setbeamercovered{transparent}
  \item O registro do paciente não deveria estar disponível a todos os funcionários.
    Por exemplo, ``um profissional de saúde não autorizado de determinada Departamento Médico poderia ver os dados
    do paciente somente para a sua especialidade e nos últimos 90 dias''.
  \item O registro do paciente deve ser anonimizado para fins de pesquisa.
  \item Acesso ao registro do paciente via web por médicos exige controle de acesso e
    criptografia.
  \item Dependência tecnológica, como por exemplo, raio-X que se deslocam via rede
    a locais remotos tornando estes sistemas críticos, pois quaisquer ataques podem
    causar danos que afetam o bem-estar do paciente.
  \end{enumerate}
  
\end{frame}


\begin{frame}{Segurança de Sistemas}{Propriedades Fundamentais da Informação}
  A Engenharia de Segurança de Sistemas tenta construir sistemas e
  estabelecer políticas de uso para garantir as seguintes propriedades
  dos dados e informações que estes manipulam:

  \begin{description}[<+-| alert@+>]\setbeamercovered{transparent}
  \item[Confidencialidade:] limita o acesso à informação às pessoas
    autorizadas pelo proprietário da informação;
  \item[Integridade:] garante manutenção das características originais
    da informação, sendo que quaisquer alterações, tais como
    manutenção e destruição, devem ser autorizadas ou realizadas pelo
    proprietário;
  \item[Disponibilidade:] a informação deve estar acessível para o uso
    legítimo, ou seja, por agentes autorizados pelo proprietário.
  \end{description}
  
\end{frame}

\begin{frame}{Aspectos da segurança da informação em sistemas}

  \begin{description}[<+-| alert@+>]\setbeamercovered{transparent}
  \item[Físico:] relaciona-se com o ambiente físico onde os dados
    estão armazenados, possíveis riscos à integridade da informação
    podem ser: incêndio, roubo da mídia de armazenamento, catástrofes
    naturais, dentre outros.
  \item[Lógico:] relaciona-se com o acesso não autorizado à informação
    utilizando um sistema computacional.
  \item[Humano:] relaciona-se com o acesso não autorizado à informação
    utilizando pessoas com acesso autorizado.
  \end{description}

\end{frame}

\begin{frame}{Mecanismos de controle}
  
  \begin{description}[<+-| alert@+>]\setbeamercovered{transparent}
  \item[Autenticação] Identificação de quem é quem, para verificar se
    a interface de gerenciamento de recursos do sistema deve ser
    oferecida;

  \item[Autorização] Verificação de quem pode o que, através do
    controle de acesso aos recursos oferecidos por um sistema.

  \item[Auditoria] Monitoramento dos registros do sistema, para 
    detecção de anomalias e tomada de providências.
  \end{description}

\end{frame}


\begin{frame}{Incidente de segurança}

\begin{quote}\footnotesize
{\bf O que é um incidente de segurança?}

Um incidente de segurança pode ser definido como qualquer evento
adverso, confirmado ou sob suspeita, relacionado à segurança de
sistemas de computação ou de redes de computadores.

São exemplos de incidentes de segurança:

\begin{itemize}[<+-| alert@+>]\setbeamercovered{transparent}
\item tentativas de ganhar acesso não autorizado a sistemas ou dados;
\item ataques de negação de serviço;
\item uso ou acesso não autorizado a um sistema;
\item modificações em um sistema, sem o conhecimento, instruções ou consentimento prévio do dono do sistema;
\item desrespeito à política de segurança ou à política de uso aceitável de uma empresa ou provedor de acesso.
\end{itemize}

\end{quote}

\hfill Extraído de \href{http://www.cert.br/docs/certbr-faq.html\#6}{FAQ: Perguntas Freqüentes ao CERT.br}

\end{frame}

\begin{frame}[allowframebreaks]{Referência}

  \begin{thebibliography}{10}
    \beamertemplatebookbibitems
  \bibitem[Ross, 2008]{ross2008}
    Ross J. Anderson.
    \newblock \href{https://www.amazon.com.br/Security-Engineering-Building-Dependable-Distributed/dp/0470068523}{``Security Engineering: A Guide to Building
      Dependable Distributed Systems''}.
    \newblock 2nd Edition, John Wiley \& Sons, 2008.

  \end{thebibliography}

\end{frame}
