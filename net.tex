\lecture{Segurança em Redes de Computadores}{net}

\lecturetitle{\course}{\insertlecture}

\frame{\maketitle}

\begin{frame}[fragile]{Setup}

  Instalar o \href{https://docs.microsoft.com/pt-br/windows/wsl/install-win10}{WSL} 
  e abrir o terminal do Linux para executar os seguintes comandos:

\begin{lstlisting}[language=bash,basicstyle=\scriptsize,numberstyle=\tiny\color{gray}]
# Suponto que a distribuição escolhida tenha sido Ubuntu ou Debian.  
sudo apt install git
git clone https://github.com/prof-holanda/auditoria-seguranca-sistemas.git
cd auditoria-seguranca-sistemas/scripts
bash install.sh net
\end{lstlisting}

\end{frame}

\begin{frame}{Referências}

   \begin{thebibliography}{10}
     \beamertemplatebookbibitems
   \bibitem[Stallings, 2014]{stallings2014}
     William Stallings,
     \newblock   \href{https://feituverava.bv3.digitalpages.com.br/users/publications/9788543005898/pages/5}{Criptografia e Segurança de Computadores. Capítulo~1}.
     \newblock Pearson, 6$^a$ edição, 2014.
     \bibitem[Holanda, 2010]{holanda2010}
     Adriano J. Holanda,
     \newblock  \href{https://speakerdeck.com/ajholanda/ldap}{Apresentação sobre protocolo LDAP}, 2010.
   \bibitem[CEHv9]{cehv9}
     \href{https://www.eccouncil.org/programs/certified-ethical-hacker-ceh/}{CEH v9 -- Certified Ethical Hacker Version 9 Study Guide},
     \newblock Sean-Philip Oriyano.
     \newblock Wiley (Digital).
   \end{thebibliography}
   
\end{frame}

\begin{frame}{Princípios Complementares}
  Como consequência da manutenção da tríade {\bf confidencialidade},
  {\bf integridade} e {\bf disponibilidade} nos sistemas, as seguintes
  propriedades são complementares:
  
  \begin{description}[<+->]\setbeamercovered{transparent}
  \item[Autenticidade:] verificação da autenticidade dos usuários, para que cada 
    entrada do sistema venha de uma fonte confiável.
  \item[Responsabilização:] o sistema deve permitir a análise pericial
    (auditoria) para identificação do responsável pelo incidente de
    segurança.
  \end{description}
\end{frame}

\frame{\title{Serviços}\date{}\author{}\maketitle}

\begin{frame}{Serviços de Segurança}

  \begin{description}[<+->]\setbeamercovered{transparent}
  \item[Autenticação:] tenta garantir a autenticidade da comunicação 
    através da identificação do proprietário da informação.
  \item[Autorização:] tenta garantir que não haja acesso não 
    autorizado a recursos. 
  \item[Auditoria:] garantir que haja detecção de eventos relevantes 
    à segurança.
  \item[Verificação de integridade:] garantir que os dados recebidos 
    são exatamente iguais aos enviados por uma entidade autorizada.
  \end{description}
  
\end{frame}

%% Métodos de autenticação??
%% Single-factor authentication
%% Two-factor authentication

\begin{frame}{Autenticação}\small
  
  Alguns protocolos e serviços que possuem módulo de autenticação são:

  \begin{itemize}[<+->]\setbeamercovered{transparent}
  \item \href{https://pt.wikipedia.org/wiki/LDAP}{LDAP} ({\em
      Lightweight Directory Access Protocol}): protocolo aberto
    (\href{http://goo.gl/IX8UPi}{RFC1487}) que fornece serviço de
    catálogo hierárquico distribuído sobre um rede IP ({\em Internet
      Protocol}). O
    \href{https://pt.wikipedia.org/wiki/Active_Directory}{\em Active
      Directory} da Microsoft implementa o LDAP.
  \item \href{https://pt.wikipedia.org/wiki/RADIUS}{RADIUS} ({\em
      Remote Authentication Dial In User Service}): protocolo de rede
    (\href{https://tools.ietf.org/html/rfc6929}{RFC6929}) para
    gerenciamento de autenticação, autorização e auditoria. Muito
    usado para fornecer acesso a recursos de rede.
  \item \href{https://en.wikipedia.org/wiki/OAuth}{OAuth}: padrão
    aberto (\href{https://tools.ietf.org/html/rfc6749}{RFC6749}) para
    autorização de acesso às informações de sites como Google,
    Facebook, Twitter por terceiros, que verifica a autenticidade do
    usuário nestes sites.
  \item
    \href{https://pt.wikipedia.org/wiki/Extensible_Authentication_Protocol}{EAP}
    ({\em Extensible Authentication Protocol}): protocolo aberto
    (\href{https://tools.ietf.org/html/rfc3748}{RFC3748}) muito
    utilizado em redes sem fio e ponto-a-ponto.
  \end{itemize}
  \end{frame}

\frame{\title{Ataques}\date{}\author{}\maketitle}

\begin{frame}{Ataques à Segurança}\small
  \begin{description}[<+->]\setbeamercovered{transparent}
  \item[Ataques passivos:] o objetivo é bisbilhotar ou monitorar as transmissões 
    para obter informação.
    \begin{description}[<+->]\setbeamercovered{transparent}
    \item[Vazamento de conteúdo:] informações sensíveis em emails, arquivos 
      ou outro meio ficam disponíveis sem a autorização do proprietário.
    \item[Análise de tráfego:] captura das mensagens para observação.
    \end{description}
  \item[Ataques ativos:] envolvem alguma modificação do fluxo de dados 
    ou criação de um fluxo falso.
    \begin{description}
    \item[Disfarce:] ocorre quando uma entidade finge ser outra, por exemplo {\it phishing}.
    \item[Modificação de mensagens:] parte da mensagem é alterada ou pacotes 
      da mensagem são reordenados para produzir efeito não-autorizado.
    \item[Negação de serviço:] impede o uso ou gerenciamento do sistema.
    \end{description}
  \end{description}
\end{frame}

  \begin{frame}{Queda de serviço}
    {\em  Denial of Service} (DoS), {\em Distributed Denial of Service} (DDoS)\\
    Ataque que visa obter um ou vários dos seguintes efeitos:
    \begin{itemize}[<+->]\setbeamercovered{transparent}
    \item Queda de performance da rede, lentidão na abertura de sites por exemplo;
    \item Indisponibilidade de um site em particular;
    \item Aumentar significativamente o número de spams recebidos;
    \item Negação de acesso por um longo tempo a um serviço de rede.
    \end{itemize}
  \end{frame}

  \begin{frame}{Queda de serviço}
    \begin{itemize}[<+->]\setbeamercovered{transparent}
    \item {\bf Enchente (flood) de ICMP (Internet Control Message
        Protocol)}: envio repetitivo de ping para um roteador mal
      configurado ou cliente com o objetivo aumentar o uso da banda e
      reduzir a qualidade de serviço.
    \item {\bf HTTP POST DDOS}: envia cabeçalho válido usando HTTP
      POST, que inclui o campo 'Content-Length' para especificar o
      corpo da mensagem a seguir, depois envia lentamente o restante,
      fazendo com que o servidor espere.
    \end{itemize}    
  \end{frame}

  \begin{frame}{Queda de serviço}{Proteção}
    Proteção:\\
    \begin{description}[<+->]\setbeamercovered{transparent}
    \item[Firewall:] o firewall pode conter regras que bloqueiam o tráfego
      de pacotes baseado em certos protocolos, portas e IPs de origem.
    \item[Switch:] alguns switches possuem capacidade de limitar o tráfego.
    \item[Roteador:] podem limitar o tráfego e controlar o acesso dos
      serviços da rede.
    \end{description}
  \end{frame}

\begin{frame}[fragile]{Prática 1}{Ataque Passivo}
  
  O \href{https://nmap.org/}{\tt nmap} é uma ferramenta para fazer varredura em
  um {\it host\/}. O exemplo a seguir mostra seu uso na interface de rede
  {\it loopback\/}:
  
\begin{lstlisting}[language=bash]
nmap localhost
\end{lstlisting}
  
\end{frame}

\begin{frame}[fragile]{Exercício 1}{DoS}
  {\color{red}$\dagger$~Vamos usar o {\tt hping3} para realizar o ataque {\em Smurf}:}
 \begin{verbatim}
 $ sudo hping3 --icmp --flood <IP|HOSTNAME> 
 $         # monitorar usando wireshark filter:icmp
 \end{verbatim}

  \bigskip\pause

{\color{blue}$\spadesuit$~A proteção é realizada fazendo bloqueio das requisições ICMP no {\em firewall}:}
\begin{verbatim}
$ sudo iptables -A OUTPUT -p icmp --icmp-type echo-request -j DROP
\end{verbatim}
  \end{frame}

  \begin{frame}[fragile]{Exercício 2}{DoS}
 {\color{red}$\dagger$~Vamos usar o {\tt ping} para realizar o {\em ping} da morte:}
\begin{verbatim}
$ sudo ping -l 65540 <IP|HOSTNAME> 
$         # monitorar usando wireshark filter:icmp e iptraf
\end{verbatim}

 \bigskip\pause

{\color{blue}$\spadesuit$~A proteção é realizada fazendo bloqueio das requisições ICMP no {\em firewall}:}
\begin{verbatim}
$ sudo iptables -A OUTPUT -p icmp --icmp-type echo-request -j DROP
\end{verbatim}
    \end{frame}


    \begin{frame}[fragile]{Exercício 3}{DoS}
    {\color{red}$\dagger$~Vamos usar o {\tt hping3} para realizar um ataque de {\sc SYN} {\em Flood} na porta 80 {\sc HTTP}:}
\begingroup\footnotesize
\begin{verbatim}
$ sudo hping3 --flood -p 80 -S <IP|HOSTNAME> 
$         # monitorar usando wireshark filter:tcp.port=80
\end{verbatim}
\endgroup

\bigskip\pause

% https://host-palace.com/knowledgebase/article/23/tcp-syn-flood-protection---iptables/
{\color{blue}$\spadesuit$~A proteção é realizada fazendo bloqueio dos pacotes SYN sem continuidade no {\em firewall} a em alguns parâmetros do {\em kernel}:}
\begingroup\footnotesize
\begin{verbatim}
$ sudo sysctl -w net.ipv4.tcp_max_syn_backlog=2048
$ sudo sysctl -w net.ipv4.tcp_syncookies=1
$ sudo sysctl -w net.ipv4.tcp_synack_retries=2
$         # bloqueia pacotes nulos
$ sudo iptables -A INPUT -p tcp --tcp-flags ALL NONE -j DROP   
$         # rejeita ataque syn-flood
$ sudo iptables -A INPUT -p tcp ! --syn -m state --state NEW -j DROP 
$         # aceita pacotes com todos estados (SYN, ACK)
$ sudo iptables -A INPUT -p tcp --tcp-flags ALL ALL -j DROP   
\end{verbatim}
\endgroup
\end{frame}

  
  \begin{frame}{Ataque de força bruta}
    \begin{itemize}[<+->]\setbeamercovered{transparent}
    \item    Tentativa de conseguir acesso a um local (blog, computador na rede, gerenciador de conteúdo) através da combinação de nomes de usuário e senhas.
    \item Envolve a exploração de uso de nomes comuns e senhas fracas e geradores de nomes e senhas
      \begin{itemize}
      \item Usuários comuns: ``admin'', ``root'', ``Administrador''
      \item Senhas fracas:
        \begin{itemize}
        \item Qualquer permutação do nome da pessoa, nome da empresa ou website;
        \item Poucos caracteres;
        \item Somente numérica ou alfanumérica;
        \item Uso de data de nascimento,  nome de parentes, animais.
        \end{itemize}
      \end{itemize}
    \end{itemize}
  \end{frame}

  \begin{frame}{Ataque de força bruta}{Proteção}
    Proteção:\\
    \begin{itemize}[<+->]\setbeamercovered{transparent}
    \item Uso de senhas fortes com caracteres numérico e alfanuméricos.
    \item Uso de bloqueadores de ataque (ex: \href{https://www.fail2ban.org/wiki/index.php/Main_Page}{Fail2ban});
    \item Restrição de acesso por IP;
    \item Mudança do número da porta padrão. Exemplo mudar a porta ssh de 22 para 2222.
    \item Remover usuários-padrão dos gerenciadores de conteúdo, fórum de discussões e blogs.
    \end{itemize}
  \end{frame}

\end{document}

== Segurança em Rede de Computadores

=== Ataques Passivos

No ataque passivo, o atacante fica monitorando e bisbilhotando a rede,
procurando por informações sensíveis provenientes de comunicações não
criptografadas. A seguir, são apresentadas algumas ferramentas que são
comumente utilizadas para este objetivo. É bom ressaltar que estas
ferramentas não foram desenvolvidas para propiciar este tipo de
ataque.

==== `nmap`

O https://nmap.org/[`nmap`] é uma ferramenta para fazer varredura em
um _host_. O exemplo a seguir mostra seu uso na interface de rede
_loopback_:

[source]
----
$ nmap localhost

Starting Nmap 7.01 ( https://nmap.org ) at 2016-09-02 09:17 BRT
Nmap scan report for localhost (127.0.0.1)
Host is up (0.00014s latency).
Not shown: 994 closed ports
PORT    STATE SERVICE
22/tcp  open  ssh
25/tcp  open  smtp
80/tcp  open  http
139/tcp open  netbios-ssn
445/tcp open  microsoft-ds
631/tcp open  ipp

Nmap done: 1 IP address (1 host up) scanned in 0.06 seconds
----

A primeira coluna mostra as portas que estão abertas para comunicação
segundo o protocolo TCP. A segunda coluna apresenta o estado de cada
porta no momento da varredura. São seis os estados possíveis, listados
a seguir:

[cols="2"]
|===
|`open`
|A porta está respondendo a uma conexão de entrada.

|`closed`
|O alvo responde à varredura, mas não há serviço aceitando conexão pela porta.

|`filtered`
|A porta está protegida por _firewall_, `nmap` não consegue descobrir se está aberta ou fechada.

|`unfiltered`
|`nmap` pode acessar, mas não consegue descobrir se está aberta ou fechada.

|`open`\|`filtered`
|`nmap` não consegue descobrir o estado exato, mas a porta parece estar aberta ou protegida por _firewall_.

|`close`\|`filtered`
|`nmap` não consegue descobrir o estado exato, mas a porta parece estar fechada ou protegida por _firewall_.
|===

Há várias formas de se utilizar o `nmap`, a seguir são mostradas algumas:

[cols="2"]
|===
|Varredura de múltiplos alvos
|`nmap 192.168.1.1 192.168.1.2 192.168.1.4`

|Varredura de uma faixa de IPs
|`nmap 192.168.1.1-100`

|Varredura de uma sub-rede
|`nmap 192.168.1.0/24`

|Fornecer um arquivo contendo os IPs a serem varridos. No exemplo, `lista.txt` contem os IPs, um por linha.
|`nmap -iL lista.txt`

|Varredura de alvos aleatórios. No exemplo, 3 _hosts_ serão escolhidos.
|`nmap -iR 3` 

|Realização uma varredura agressiva.
|`nmap -A 10.10.4.132`

|Varredura usando protocolo IPv6.
|`nmap -6 0:0:0:0:0:ffff:c0a8:101`

|Varredura com detecção do sistema operacional.
|`nmap -O 192.168.1.32`

|Saída da varredura com mais informação.
|`nmap -v 192.168.2.12`

|Saída com _debug_.
|`nmap -d 102.168.1.43`
|===

Uma boa referência de consulta é o livro http://goo.gl/QvgUa2["_Nmap_ 6 _Cookbook_"] escrito por Nicholas Marsh.

==== `tcpdump`

O http://www.tcpdump.org/[`tcpdump`] é ferramenta de análise de
pacotes TCP/IP, que pode ser usada para monitorar tráfego de
informações sensíveis. A seguir, são mostrados alguns exemplos de seu
uso, lembrando que o usuário deve ter privilégios de admistrador para
poder executá-lo:

[cols="2"]
|===

|Uso básico.
|`tcpdump -nS`

|Uso básico com mais informação.
|`tcpdump -nnvvS`

|Saída com ainda mais informação.
|`tcpdump -nnvvXS`

|Restrição do tráfego baseada no IP.
|`tcpdump host 192.168.1.178`

|Seleciona o tráfego de uma rede para analizar.
|`tcpdump net 192.168.1.0/24`

|Seleciona os pacotes por protocolo.
|`tcpdump icmp`

|Seleciona os pacotes por porta.
|`tcpdump port 80`

|Seleciona os pacotes de acordo com a direção do fluxo e porta, onde `src` é a fonte do pacote e `dst` é o destino.
|`tcpdump src port 80 and dst port 21`

|Combinação de direção, porta e protocolo.
|`tcpdump src port 1025 and tcp`

|Especificação de uma faixa de portas.
|`tcpdump portrange 21-23`

|Filtro pelo tamanho do pacote. Podem ser usados palavras ou símbolos como `less`, `<`, `greater`, `>=`.
|`tcpdump < 32`

|Captura do tráfego da porta 80 para o arquivo `captura.txt`.
|`tcpdump -s 1514 port 80 -w captura.txt`

|Captura com exceção da porta 80.
|`tcpdump not port 80`

|Pacotes cuja fonte é o IP 192.168.1.23 e as portas de destino são 389 ou 636.
|`tcpdump src 192.168.1.23 and (dst port 380 or 636)`

|Mostra  todos os pacotes urgentes (URG).
|tcpdump 'tcp[13] & 32!=0'

|Mostra  todos os pacotes no estado ACK.
|tcpdump 'tcp[13] & 16!=0'

|Mostra pacotes no estado SYN.
|tcpdump 'tcp[13] & 2!=0'

|Mostra pacotes no estado FIN.
|tcpdump 'tcp[13] & 1!=0'

|Mostra pacotes no estado SYNACK.
|tcpdump 'tcp[13]=18'
|===

A fonte do material apresentado foi o artigo
https://danielmiessler.com/study/tcpdump/["A tcpdump Tutorial and
Primer with Examples"] de Daniel Miessler.

=== Ataques Ativos

Os ataques ativos podem ser prevenidos da seguinte forma:

* *Disfarce* (_phishing_): uma pessoa se passa por outra, utilizando
senha, cartão, ou qualquer outro meio de indentificação que é obtido
normalmente através de engenharia social ou _malware_. É difícil de
ser prevenido via sistema, pois depende do entendimento da pessoa
sobre segurança, para não repassar ou deixar disponível informação
sensível.

* *Modificação de mensagens*: Algumas regras podem ser colocadas no
_firewall_ para prevenir que pacotes corrompidos sejam entregues.

* *Negação de serviço*: Ataques de negação de serviços podem ser
prevenidos através da configuração adequada do _firewall_.
// TODO fazer ligação com o capítulo de firewall.

////
=== Serviços de segurança

==== Autenticação

==== Autorização

==== Verificação de Integridade

////
