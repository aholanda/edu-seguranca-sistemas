\lstset{language=bash}

\section*{Firewall}

\subsection*{iptables}

Antes de construir quaisquer regras, vamos apagar regras que já existam:


\begin{lstlisting}
  sudo iptables -F
\end{lstlisting}

Para listar as regras existentes, teclamos

\begin{lstlisting}
    sudo iptables -L
 \end{lstlisting}
  
Após a limpeza das regras, a listagem mostra

\begin{lstlisting}
    Chain INPUT (policy ACCEPT)
    target     prot opt source               destination         

    Chain FORWARD (policy ACCEPT)
    target     prot opt source               destination         

    Chain OUTPUT (policy ACCEPT)
    target     prot opt source               destination
  \end{lstlisting}
  
A política padrão é `ACCEPT`, ou seja, não há qualquer tipo de filtro
para as cadeias de entrada `INPUT`, saída `OUTPUT` e repasse `FOWARD`
de pacotes.

Vamos alterar a política padrão para derrubar todos os pacotes nos 3
canais:

\begin{lstlisting}
    sudo iptables -P INPUT DROP
    sudo iptables -P FORWARD DROP
    sudo iptables -P OUTPUT DROP
  \end{lstlisting}
  
Quando executamos

\begin{lstlisting}
    ping -c 3 127.0.0.1
  \end{lstlisting}
  
\noindent a saída mostra o bloqueio do loopback

\begin{lstlisting}
    ping: sendmsg: Operation not permitted
    ping: sendmsg: Operation not permitted
    ping: sendmsg: Operation not permitted
  \end{lstlisting}
  
Vamos liberar o acesso ao *loopback*

\begin{lstlisting}
    sudo iptables -A INPUT -i lo -j ACCEPT
    sudo iptables -A OUTPUT -o lo -j ACCEPT
  \end{lstlisting}
  
\noindent e repetir o ping

\begin{lstlisting}
    ping -c 3 127.0.0.1
  \end{lstlisting}
  
\noindent e obtemos algo parecido com

\begin{lstlisting}
    PING 127.0.0.1 (127.0.0.1) 56(84) bytes of data.
    64 bytes from 127.0.0.1: icmp_seq=1 ttl=64 time=0.027 ms
    64 bytes from 127.0.0.1: icmp_seq=2 ttl=64 time=0.027 ms
    64 bytes from 127.0.0.1: icmp_seq=3 ttl=64 time=0.033 ms

    --- 127.0.0.1 ping statistics ---
    3 packets transmitted, 3 received, 0% packet loss, time 1998ms
    rtt min/avg/max/mdev = 0.027/0.029/0.033/0.003 ms
  \end{lstlisting}
  
Porém, não conseguimos enviar o ping para fora do firewall, pois

\begin{lstlisting}
    ping -c 3 google.com
  \end{lstlisting}
  
\noindent mostra que o acesso ao servidor de DNS (porta 53) está bloqueado, 
vamos liberar a porta 53

\begin{lstlisting}
    sudo iptables -A OUTPUT -p udp -o eth0 --dport 53 -j ACCEPT
    sudo iptables -A INPUT -p udp -i eth0 --sport 53 -j ACCEPT
  \end{lstlisting}
  
e executar novamente

\begin{lstlisting}
    ping -c 3 google.com
  \end{lstlisting}
  
\noindent cuja saída mostra que os pacotes `ICMP` continuam bloqueados

\begin{lstlisting}
    PING google.com (XXX.XXX.XXX.XXX) 56(84) bytes of data.
    ping: sendmsg: Operation not permitted
    ping: sendmsg: Operation not permitted
    ping: sendmsg: Operation not permitted

    --- google.com ping statistics ---
    3 packets transmitted, 0 received, 100% packet loss, time 1999ms
  \end{lstlisting}
  
Vamos liberar o ping de dentro para fora do firewall


\begin{lstlisting}
    sudo iptables -A OUTPUT -p icmp --icmp-type echo-request -j ACCEPT
    sudo iptables -A INPUT -p icmp --icmp-type echo-reply -j ACCEPT
\end{lstlisting}

\noindent e novamente

\begin{lstlisting}
    ping -c 3 holanda.xyz
\end{lstlisting}

\noindent obtemos


\begin{lstlisting}
Disparando google.com [XXX:XXXX:XXXX:XXX::XXXX] com 32 bytes de dados:
   
Resposta de XXX:XXXX:XXXX:XXX::XXXX: tempo=11ms 
Resposta de XXX:XXXX:XXXX:XXX::XXXX: tempo=9ms 
Resposta de XXX:XXXX:XXXX:XXX::XXXX: tempo=10ms 
Resposta de XXX:XXXX:XXXX:XXX::XXXX: tempo=11ms 


Estatísticas do Ping para XXX:XXXX:XXXX:XXX::XXXX:
    Pacotes: Enviados = 4, Recebidos = 4, Perdidos = 0 (0% de
             perda),
Aproximar um número redondo de vezes em milissegundos:
    Mínimo = 9ms, Máximo = 11ms, Média = 10ms
\end{lstlisting}

\noindent mostando que o ICMP agora está liberado.


Porém ainda não conseguimos navegar na web, vamos liberar a porta `HTTP` (80) 
 e `HTTPS` (443)


\begin{lstlisting}
    sudo iptables -A OUTPUT -p tcp -o eth0 --dport 80 -j ACCEPT
    sudo iptables -A INPUT -p tcp -i eth0 --sport 80 -j ACCEPT
    sudo iptables -A OUTPUT -p tcp -o eth0 --dport 443 -j ACCEPT
    sudo iptables -A INPUT -p tcp -i eth0 --sport 443 -j ACCEPT
\end{lstlisting}

\subsection*{Exercícios}
\begin{enumerate}
\item Qual o significado dos argumentos {\tt NEW}, {\tt ESTABLISHED} e
{\tt RELATED} para a {\em flag} {\tt -state} na seguinte regra:

\begin{lstlisting}
sudo iptables --append OUTPUT -m state \
  --state NEW,ESTABLISHED,RELATED -j ACCEPT
\end{lstlisting}

\item Escreva uma regra que libere a comunicação via usando {\tt ssh} (porta 22).
\end{enumerate}
