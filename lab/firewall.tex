\lstset{style=mybash}

\section*{Firewall}

\rootwarn{firewall}{nftables}

\subsection*{nftables}

Nas distribuições derivadas do Debian, o caminho do arquivo de de
configuração é {\tt /etc/nftables.conf}. Para listar as tabelas
existentes já configuradas neste arquivo, executamos o comando

\begin{lstlisting}
sudo nft list tables
\end{lstlisting}
  
\noindent cuja saída antes de qualquer configuração normalmente é

\begin{lstlisting}
sudo table inet filter
\end{lstlisting}

Na tabela {\tt filter} da família {\tt inet} são adicionadas as regras
para cada cadeia ({\em chain}). Executamos o comando

\begin{lstlisting}
sudo nft list table inet filter
\end{lstlisting}

\noindent para listar as regras de cada cadeia na tabela {\tt inet
  filter}:

\begin{lstlisting}
table inet filter {
        chain input {
                type filter hook input priority filter; policy accept;
        }

        chain forward {
                type filter hook forward priority filter; policy accept;
        }

        chain output {
                type filter hook output priority filter; policy accept;
        }
}
\end{lstlisting}

A política padrão é {\tt accept}, ou seja, não há qualquer tipo de
filtro para as cadeias de entrada {\tt input}, saída {\tt output} e
repasse {\tt foward} de pacotes.

Vamos alterar a política padrão para derrubar todos os pacotes nas 3
cadeias:

\begin{lstlisting}
sudo nft add chain inet filter input '{ policy drop; }'
sudo nft add chain inet filter forward '{ policy drop; }'
sudo nft add chain inet filter output '{ policy drop; }'
\end{lstlisting}
  
Vamos testar a política padrão configurada que bloqueia qualquer
entrada, saída ou repasse de pacotes de rede:

\begin{lstlisting}
ping -c 3 127.0.0.1
\end{lstlisting}
  
\noindent a saída mostra o bloqueio da interface {\it loopback\/}.

\begin{lstlisting}
PING 127.0.0.1 (127.0.0.1) 56(84) bytes of data.

--- 127.0.0.1 ping statistics ---
18 packets transmitted, 0 received, 100% packet loss, time 17406ms
\end{lstlisting}
  
Supondo que vamos adotar uma política de filtragem bem restritiva,
começamos a liberar o fluxo de pacotes nas interfaces e serviços de
interesse. Vamos começar liberando o protocolo ICMP na entrada e
saída:

\begin{lstlisting}
sudo nft add rule inet filter input ip protocol icmp accept  
sudo nft add rule inet filter output ip protocol icmp accept
\end{lstlisting}
  
\noindent e repetir o ping
  
\noindent e obtemos algo parecido com

\begin{lstlisting}
PING 127.0.0.1 (127.0.0.1) 56(84) bytes of data.
64 bytes from 127.0.0.1: icmp_seq=1 ttl=64 time=0.022 ms
64 bytes from 127.0.0.1: icmp_seq=2 ttl=64 time=0.051 ms
64 bytes from 127.0.0.1: icmp_seq=3 ttl=64 time=0.035 ms

--- 127.0.0.1 ping statistics ---
3 packets transmitted, 3 received, 0% packet loss, time 2040ms
rtt min/avg/max/mdev = 0.022/0.036/0.051/0.011 ms
\end{lstlisting}
  
Porém, não conseguimos enviar o ping para fora do firewall usando o nome
do host, apenas o IP:

\begin{lstlisting}
ping -c 3 google.com
\end{lstlisting}
  
\noindent mostra que o acesso ao servidor de DNS (porta 53) está bloqueado, 
vamos liberar a porta 53

\begin{lstlisting}
sudo nft add rule inet filter output udp dport 53 accept
sudo nft add rule inet filter output tcp dport 53 accept
\end{lstlisting}
  
e executar novamente

\begin{lstlisting}
ping -c 3 google.com
\end{lstlisting}
  
\noindent cuja saída mostra que os pacotes {\tt ICMP} continuam
bloqueados, pois devemos liberar as portas que são usadas pela
resposta do {\tt ping}, então estabelecemos uma regra de que para
comexões estabelecidas, as portas sejam liberadas.

\begin{lstlisting}
sudo nft add rule inet filter input ct state established,related counter accept  
sudo nft add rule inet filter output ct state established,related counter accept
\end{lstlisting}
  
\noindent e novamente

\begin{lstlisting}
ping -c 3 google.com
\end{lstlisting}

\noindent obtemos

\begin{lstlisting}
PING google.com (xxx.xxx.xxx.xxx) 56(84) bytes of data.
64 bytes from gruxxx.net (xxx.xxx.xxx.xxx): icmp_seq=1 ttl=117 time=8
.56 ms
64 bytes from gruxxx.net (xxx.xxx.xxx.xxx): icmp_seq=2 ttl=117 time=8
.48 ms
64 bytes from gruxxx.net (xxx.xxx.xxx.xxx): icmp_seq=3 ttl=117 time=8
.52 ms

--- google.com ping statistics ---
3 packets transmitted, 3 received, 0% packet loss, time 2002ms
rtt min/avg/max/mdev = 8.483/8.520/8.559/0.031 ms
\end{lstlisting}

\noindent mostando que o ICMP e DNS (saída) agora estão em operação.

Porém ainda não conseguimos navegar na web, vamos liberar as portas
HTTP (80) e HTTPS (443) na cadeia de saída:

\begin{lstlisting}
sudo nft add rule inet filter output tcp dport '{80,443}' accept
\end{lstlisting}

\subsection*{Exercícios}
\begin{enumerate}
\item Qual a configuração do {\tt nftables} para liberar a comunicação
  de um servidor {\tt ssh}?

\item Supondo que queiramos instalar um servidor de emails
  em nosso host, liste as regras para liberar as portas
  dos protocolos SMTP e IMAP.
\end{enumerate}
