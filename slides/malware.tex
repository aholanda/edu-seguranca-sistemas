\lecture{Malware}{malware}
\lecturetitle{\insertlecture}{\course}

\frame{\maketitle}

\begin{frame}{\insertlecture}\small
  
  {\it \insertlecture} ({\it {\bf mal}icious soft{\bf ware}\/}) é qualquer programa que modifica 
  o funcionamento correto do sistema operacional, acessa informação sensível ou 
  mostra avisos não desejados. \pause Dentre os tipos, podemos listar:

  \begin{itemize}
  \item Vírus;
  \item {\it Worms\/} (Vermes);
  \item Cavalo de troia ({\it trojan horse\/});
  \item {\it Spyware\/};
  \item {\it Adware\/};
  \item {\it Backdoor\/};
  \item {\it Rootkit\/};
  \item {\it Ransomware\/};
  \item {\it Botnet\/}.
  \end{itemize}
\end{frame}

\begin{frame}{Vírus}
\begin{itemize}[<+->]\setbeamercovered{transparent}
\item Vírus de computador que quando executado, replica a si próprio
  em outros programas, arquivos de dados, setor de boot do disco
  rígido.
\item Os vírus provocam danos ao computador infectado tais como: roubo
  de espaço em disco, tempo de processador, acesso à informação
  privada, corrupção dos dados, captura de eventos de teclado.
\end{itemize}

\pause\bigskip
\alert{Proteção: anti-vírus}
\end{frame}

\begin{frame}{\em Worm}{Verme}

  \begin{itemize}[<+->]\setbeamercovered{transparent}
    \item Os {\em worms} são parecidos com os vírus, com a diferença de que eles 
      se replicam sem a necessidade de infectar outros programas, ou seja, 
      são programas completos.
    \item Os danos provocados são os mesmos do vírus e a proteção também é a 
      mesma.
    \end{itemize}
\end{frame}

\begin{frame}{Cavalo de troia}{\it Trojan horse}\footnotesize
  Programa que parece ser útil e convence o usuário a instalá-lo,
  porém, quando executado causa sérios danos, tais como:
  
  \begin{itemize}
    \item Modificação ou remoção de arquivos;
    \item Corrupção dos dados;
    \item Formatação de disco;
    \item Espalha vírus pela internet;
    \item Envia spam;
    \item Captura eventos de teclado e vídeo;
    \item Transfere o controle do computador para um usuário remoto.
    \end{itemize}

    \pause\bigskip
    \alert{Proteção:}
    \begin{itemize}
    \item Consulta a boletins de segurança antes de instalar programas desconhecidos;
    \item Antivírus.
    \end{itemize}
  \end{frame}

\begin{frame}{Outros \insertlecture s}\footnotesize
  \begin{description}[<+->]\setbeamercovered{transparent}
  \item[{\em Spyware}:] programa que recolhe informações do computador 
    sem a autorização do usuário, transmitindo esta informação via rede.
  \item[{\em Adware}:] programa que executa automaticamente e exibe 
    grande quantidade de anúncios sem a permissão do usuário.
  \item[{\em Backdoor}:] programas que conseguem acesso ao sistema ou rede 
    explorando falhas críticas de sistemas ou brechas usadas por desenvolvedores 
    para testar o sistema.
  \item[{\em Rootkit}:] programa que altera o comportamento do sistema operacional 
    para esconder a existência de processos maliciosos ou não autorizados pelo 
    usuário.
  \item[{\em Ransomware}:] programa que restringe o acesso aos dados ou programa 
    infectado e o atacante cobra um valor para que o usuário tenha de volta 
    o estado normal de funcionamento.
  \item[{\em Botnet}:] coleção de programas (agentes) que se comunicam para 
    executar ações em conjunto. 
  \end{description}  
\end{frame}
